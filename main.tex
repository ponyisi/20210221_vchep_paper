%%%%%%%%%%%%%%%%%%%%%%% file template.tex %%%%%%%%%%%%%%%%%%%%%%%%%
%
% This is a template file for Web of Conferences Journal
%
% Copy it to a new file with a new name and use it as the basis
% for your article
%
%%%%%%%%%%%%%%%%%%%%%%%%%% EDP Science %%%%%%%%%%%%%%%%%%%%%%%%%%%%
%
%%%\documentclass[option]{webofc}
%%% "twocolumn" for typesetting an article in two columns format (default one column)
%
\documentclass{webofc}
\usepackage[varg]{txfonts}   % Web of Conferences font
%
% Put here some packages required or/and some personnal commands
%
%
\begin{document}
%
\title{Recent Improvements to the ATLAS Data Quality Monitoring System}
%
% subtitle is optionnal
%
%%%\subtitle{Do you have a subtitle?\\ If so, write it here}

\author{\firstname{Peter} \lastname{Onyisi}\inst{1}\fnsep\thanks{\email{Mail address for first
    author}} \and
        \firstname{Charles} \lastname{Burton}\inst{1} \and
        \firstname{Third} \lastname{Author}\inst{3}
        % etc.
}

\institute{University of Texas at Austin
\and
           the second here 
          }

\abstract{%
  Insert your english abstract here.
}
%
\maketitle
%
\section{Introduction}
\label{intro}
During the second Long Shutdown (LS2) of the Large Hadron Collider (LHC), the ATLAS experiment has updated various aspects of its offline data quality monitoring software. This paper reviews key changes that have been implemented, with a focus on tools and techniques with potential impact outside of the ATLAS collaboration.
\section{Multithreaded Histogramming in Reconstruction}

\section{Histogram Postprocessing}
It is frequently necessary to perform a processing step on a collected histogram in order to compute a desired result (for example, computing the mean of position residuals to detect problems with detector alignment). To enable multithread-safe filling of histograms, we require that histogram handling in reconstruction software consists entirely of accumulation operations. This implies that merging of histograms is a straightforward operation which merely 

In other contexts this is part of the \textit{map-reduce} paradigm, in which the map step is the collection of the histograms

\section{User Presentation}

\section{Data Quality Result Storage}
\label{sec-1}
For bibliography use \cite{RefJ}
\subsection{Subsection title}
\label{sec-2}
Don't forget to give each section, subsection, subsubsection, and
paragraph a unique label (see Sect.~\ref{sec-1}).

For one-column wide figures use syntax of figure~\ref{fig-1}
\begin{figure}[h]
% Use the relevant command for your figure-insertion program
% to insert the figure file.
\centering
\includegraphics[width=1cm,clip]{tiger}
\caption{Please write your figure caption here}
\label{fig-1}       % Give a unique label
\end{figure}

For two-column wide figures use syntax of figure~\ref{fig-2}
\begin{figure*}
\centering
% Use the relevant command for your figure-insertion program
% to insert the figure file. See example above.
% If not, use
\vspace*{5cm}       % Give the correct figure height in cm
\caption{Please write your figure caption here}
\label{fig-2}       % Give a unique label
\end{figure*}

For figure with sidecaption legend use syntax of figure
\begin{figure}
% Use the relevant command for your figure-insertion program
% to insert the figure file.
\centering
\sidecaption
\includegraphics[width=5cm,clip]{tiger}
\caption{Please write your figure caption here}
\label{fig-3}       % Give a unique label
\end{figure}

For tables use syntax in table~\ref{tab-1}.
\begin{table}
\centering
\caption{Please write your table caption here}
\label{tab-1}       % Give a unique label
% For LaTeX tables you can use
\begin{tabular}{lll}
\hline
first & second & third  \\\hline
number & number & number \\
number & number & number \\\hline
\end{tabular}
% Or use
\vspace*{5cm}  % with the correct table height
\end{table}
%
% BibTeX or Biber users please use (the style is already called in the class, ensure that the "woc.bst" style is in your local directory)
% \bibliography{name or your bibliography database}
%
% Non-BibTeX users please use
%
\begin{thebibliography}{}
%
% and use \bibitem to create references.
%
\bibitem{RefJ}
% Format for Journal Reference
Journal Author, Journal \textbf{Volume}, page numbers (year)
% Format for books
\bibitem{RefB}
Book Author, \textit{Book title} (Publisher, place, year) page numbers
% etc
\end{thebibliography}

\end{document}

% end of file template.tex